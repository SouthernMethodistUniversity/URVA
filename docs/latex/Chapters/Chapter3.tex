
\chapter{Output description}

In pURVA, results are all written to external files instead of standard output. All output files have the suffix of ``.csv'' or ``.dat''. The execution of pURVA will abort if a result file with a duplicated name is found in the current folder. Make sure that current directory is cleaned up before execution.


\section{Energy and derivatives}

Usually the Self-Consistent Field(SCF) energy is calculated and used to construct the potential energy surface along the reaction path.

In order to check this value, the \texttt{\textcolor{red}{@DATAFILETYPE}} must be set to \texttt{\textcolor{blue}{old}} or \texttt{\textcolor{blue}{new}} and \texttt{\textcolor{red}{@ENERGY}} must be set to \texttt{\textcolor{blue}{on}}.

The unit of SCF energy is Hartree as one of the atomic units(a.u.). In order to calculate first and second derivatives of SCF energy against reaction coordinate/parameter $\mathbf{s}$, cubic spline fitting is used. For the second derivative of SCF energy, the region between $\mathbf{s}=-0.1$ and $\mathbf{s}=+0.1$ is predicted via cubic spline fitting from the information outside this region.

\textit{NOTE: 1 Hartree = 627.509 474 kcal/mol}

Output files:
\begin{itemize}
    \item \texttt{\textcolor{blue}{energy.csv}}
    
    SCF energy vs. $\mathbf{s}$
    
    \item \texttt{\textcolor{blue}{energy\_1\_d.csv}}
    
    First derivative of SCF energy vs. $\mathbf{s}$
    
    \item \texttt{\textcolor{blue}{energy\_2\_d.csv}}    
    
    Second derivative of SCF energy vs. $\mathbf{s}$    
    
\end{itemize}



\section{Internal coordinates}
The value of user-defined internal coordinates could be calculated. All types of internal coordinates described in section 2.2.2 are supported. 

In order to have this result, \texttt{\textcolor{red}{@PARM}} must be set to \texttt{\textcolor{blue}{GeomOnly}} or \texttt{\textcolor{blue}{All}}.

The unit of printed internal coordinates is atomic unit with bohr for distance and radian for angles.

\textit{NOTE: 1 Bohr = 0.529177 $\AA$; 1 rad = 57.295 8 $^{\circ}$ }


Output file:
\begin{itemize}
    \item \texttt{\textcolor{blue}{q$\_$n.csv}}
\end{itemize}






\section{Decomposition of reaction path direction and curvature into internal coordinates}


In order to have this result, \texttt{\textcolor{red}{@PARM}} must be set to  \texttt{\textcolor{blue}{All}}.

Output files:
\begin{itemize}
    \item \texttt{\textcolor{blue}{eta-q\_n.csv}}
    
    Decomposition of reaction path direction into internal coordinates
    
    \item \texttt{\textcolor{blue}{kappa-q\_n.csv}}
    
    Decomposition of reaction path curvature into internal coordinates
    
\end{itemize}


\section{Generalized vibrational frequency}
For any point on reaction path, we could have $3N - K - 1$ vibrations, in which $K$ is the total number of translations and rotations. In pURVA, $K$ takes the value of 6 which excludes the possibility of analysis of reactions like H$_2$ + H $\rightarrow$ H + H$_2$ where the whole reaction complex stays in a linear geometry.

In order to have this result, \texttt{\textcolor{red}{@VIBRATION}} must be set to  \texttt{\textcolor{blue}{on}}.

Output file:
\begin{itemize}
    \item \texttt{\textcolor{blue}{freq\_dmo.csv}}
    
    Generalized vibrational frequencies vs. $\mathbf{s}$

\end{itemize}

\textit{NOTE: Unit of frequencies is $cm^{-1}$.}


\section{Scalar curvature}

The original scalar curvature calculated without correction around the TS region and spike removal will be written to file \texttt{\textcolor{blue}{originalkappa.dat}}.

In order to have this result, \texttt{\textcolor{red}{@DIRCURV}} must be set to  \texttt{\textcolor{blue}{on}}.


If the CURVCOR and AUTOSMTH modules are used, the corrected curvature data will be written to \texttt{\textcolor{blue}{merged.dat}}.

If RMSPK module is also used, the curvature data after spike removal will be written to \texttt{\textcolor{blue}{merged-nospk.dat}}.




\section{Adiabatic force constant}
The adiabatic force constant of chemical bonds between two atoms along the reaction path will be written to \texttt{\textcolor{blue}{adiabfc-ka.csv}}.

In order to have this result, \texttt{\textcolor{red}{@ADIABFC}} must be set to  \texttt{\textcolor{blue}{on}}.


In some situations, there might be noise in the result. These noise regions could be nicely removed via cubic spline fitting.

\textit{NOTE: Only result of bond length between 2 atoms could make sense.}




