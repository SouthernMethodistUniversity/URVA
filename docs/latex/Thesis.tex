%% ----------------------------------------------------------------
%% Thesis.tex -- MAIN FILE (the one that you compile with LaTeX)
%% ---------------------------------------------------------------- 

% Set up the document
\documentclass[a4paper, 11pt, oneside]{Thesis}  % Use the "Thesis" style, based on the ECS Thesis style by Steve Gunn
\graphicspath{Figures/}  % Location of the graphics files (set up for graphics to be in PDF format)

% Include any extra LaTeX packages required
\usepackage[square, numbers, comma, sort&compress]{natbib}  % Use the "Natbib" style for the references in the Bibliography
\usepackage{verbatim}  % Needed for the "comment" environment to make LaTeX comments
\usepackage{vector}  % Allows "\bvec{}" and "\buvec{}" for "blackboard" style bold vectors in maths
\usepackage{courier}
\usepackage{amsmath}
\usepackage[usenames, dvipsnames]{color}
\usepackage{tcolorbox}



\hypersetup{urlcolor=blue, colorlinks=true}  % Colours hyperlinks in blue, but this can be distracting if there are many links.

%% ----------------------------------------------------------------
\begin{document}
\frontmatter      % Begin Roman style (i, ii, iii, iv...) page numbering

% Set up the Title Page
\title  {User Manual for pURVA program}
\authors  {\texorpdfstring
            {\href{ywtao.smu@gmail.com}{Yunwen Tao}}
            {Yunwen Tao}
            }
\addresses  {\groupname\\\deptname\\\univname}  % Do not change this here, instead these must be set in the "Thesis.cls" file, please look through it instead
\date       {\today}
\subject    {}
\keywords   {}

\maketitle
%% ----------------------------------------------------------------

\setstretch{1.3}  % It is better to have smaller font and larger line spacing than the other way round

% Define the page headers using the FancyHdr package and set up for one-sided printing
\fancyhead{}  % Clears all page headers and footers
\rhead{\thepage}  % Sets the right side header to show the page number
\lhead{}  % Clears the left side page header

\pagestyle{fancy}  % Finally, use the "fancy" page style to implement the FancyHdr headers

%% ----------------------------------------------------------------
% Declaration Page required for the Thesis, your institution may give you a different text to place here

\clearpage  % Declaration ended, now start a new page


%The Abstract Page
%\addtotoc{Abstract}  % Add the "Abstract" page entry to the Contents
%\abstract{
%\addtocontents{toc}{\vspace{1em}}  % Add a gap in the Contents, for aesthetics

%The Thesis Abstract is written here (and usually kept to just this page). The page is %kept centered vertically so can expand into the blank space above the title too\ldots

%}

%\clearpage  % Abstract ended, start a new page
%% ----------------------------------------------------------------

\setstretch{1.3}  % Reset the line-spacing to 1.3 for body text (if it has changed)

% The Acknowledgements page, for thanking everyone
%\acknowledgements{
%\addtocontents{toc}{\vspace{1em}}  % Add a gap in the Contents, for aesthetics

%The acknowledgements and the people to thank go here, %don't forget to include your project advisor\ldots

%}
%\clearpage  % End of the Acknowledgements
%% ----------------------------------------------------------------

\pagestyle{fancy}  %The page style headers have been "empty" all this time, now use the "fancy" headers as defined before to bring them back


%% ----------------------------------------------------------------
\lhead{\emph{Contents}}  % Set the left side page header to "Contents"
\tableofcontents  % Write out the Table of Contents

%% ----------------------------------------------------------------
%\lhead{\emph{List of Figures}}  % Set the left side page header to "List if Figures"
%\listoffigures  % Write out the List of Figures

%% ----------------------------------------------------------------
%\lhead{\emph{List of Tables}}  % Set the left side page header to "List of Tables"
%\listoftables  % Write out the List of Tables

%% ----------------------------------------------------------------
\setstretch{1.5}  % Set the line spacing to 1.5, this makes the following tables easier to read
\clearpage  % Start a new page
\lhead{\emph{Abbreviations}}  % Set the left side page header to "Abbreviations"
\listofsymbols{ll}  % Include a list of Abbreviations (a table of two columns)
{
% \textbf{Acronym} & \textbf{W}hat (it) \textbf{S}tands \textbf{F}or \\
\textbf{pURVA} & standalone \textbf{p}rogram of \textbf{U}nified \textbf{R}eaction \textbf{V}alley \textbf{A}pproach \\

\textbf{IRC} & \textbf{I}ntrinsic \textbf{R}eaction \textbf{C}oordinates


}


%% ----------------------------------------------------------------
\clearpage  % Start a new page
%\lhead{\emph{Physical Constants}}  % Set the left side page header to "Physical Constants"
%\listofconstants{lrcl}  % Include a list of Physical Constants (a four column table)
%{
% Constant Name & Symbol & = & Constant Value (with units) \\
%Speed of Light & $c$ & $=$ & $2.997\ 924\ 58\times10^{8}\ \mbox{ms}^{-\mbox{s}}$ %(exact)\\

%}

%% ----------------------------------------------------------------
%\clearpage  %Start a new page
\lhead{\emph{Symbols}}  % Set the left side page header to "Symbols"
\listofnomenclature{lll}  % Include a list of Symbols (a three column table)
{
% symbol & name & unit \\
$\AA$ & distance & Angstrom \\
%$P$ & power & W (Js$^{-1}$) \\
%& & \\ % Gap to separate the Roman symbols from the Greek
$\omega$ & harmonic frequency & cm$^{-1}$ 
}
%% ----------------------------------------------------------------
% End of the pre-able, contents and lists of things
% Begin the Dedication page

\setstretch{1.3}  % Return the line spacing back to 1.3

\pagestyle{empty}  % Page style needs to be empty for this page
%\dedicatory{For/Dedicated to/To my\ldots}

\addtocontents{toc}{\vspace{2em}}  % Add a gap in the Contents, for aesthetics


%% ----------------------------------------------------------------
\mainmatter	  % Begin normal, numeric (1,2,3...) page numbering
\pagestyle{fancy}  % Return the page headers back to the "fancy" style

% Include the chapters of the thesis, as separate files
% Just uncomment the lines as you write the chapters

\lhead{\emph{Chapter 1}}

\chapter{Before we start}

It is not a good idea to have no idea about the basic copyright, history, requirement, functionality as well as theory of the program before we use it.

\section{Copyright}

The program pURVA as well as this manual MUST NOT be released out of CATCO group at Southern Methodist University. According to policy \textsection12.1 and \textsection12.2 of Southern Methodist University, the leakage of the intellectually property may face legal charge.

\section{History}

The original URVA method was implemented by Zoran Konkoli in the link L716 of Gaussian package. However, this part is never incorporated into the public version of Gaussian. Since then, many other contributors including Dr. Wenli Zou added functionalities into this part and migrated this part from older version of Gaussian into newer version of Gaussian for several times. As Gaussian package was written in Fortran 77, the corresponding URVA part was written in the same language.

Later on from 2015, Dr. Dieter Cremer and Dr. Elfi Kraka wanted to have an independent version of URVA program. They asked Yunwen Tao in the group to this job. He started with programming in Fortran 90 language which is an extension to Fortran 77. Then he switched the whole project into Python language which is more flexible and easy to use. The new version of the URVA program was then named as \textbf{pURVA}. 



\section{Execution of pURVA }
The proper execution of pURVA requires Python interpreter with the version 2.7.x. Versions lower than this might lead to trouble.

Here are the list of Python modules needed to run pURVA: (1) NumPy, (2) SciPy, (3) SymPy, (4) sys, (5) os, (6) copy,  (7) gc, (8) math and (9) time.

Make sure that all these modules have been installed properly.



pURVA expects and then reads in an external text file as the user input file. After this file is prepared, in the terminal, type in 

\texttt{\$ python main.py \textit{myinputfile}}

From the standard output, we could monitor how the calculation goes. The calculation results will be dumped into external text files on the disk.

To make life easier, running pURVA on ManeFrame cluster is recommended as pURVA has been developed and tested on the same machine. Before running pURVA, remember to load the Python interpreter by using 

\texttt{\$ module load python}


\section{Theory as Unified Reaction Valley Approach}
The name of ``Unified Reaction Valley Approach'' firstly appeared on scientific journals in 1997 when Konkoli, Kraka and Cremer published their comprehensive studies of CH$_3$ + H$_2$ $\rightarrow$ CH$_4$ + H on \textit{J. Phys. Chem. A}\cite{firsturva}. In that paper, one of the highlights is to introduce the approach that calculates the adiabatic mode coupling coefficient that is decomposition of reaction path curvature into adiabatic local modes which was a novel approach dealing with vibrational spectroscopy. URVA is based on the Reaction Path Hamiltonian(RPH) that was intensively developed by Miller, Handy\cite{MillerHandy}, Page and McIver\cite{PageMcIver}. In the year of 2011, Dr. Kraka published a well-written review on the relationship between RPH and URVA\cite{Krakareview}. Most recently, Dr. Zou proposed a new approach to decompose the reaction path direction and curvature into internal coordinates which opens the possibility to study chemical reactions in large systems, e.g. organometallic compounds and enzymes\cite{Zou2016}.  

One of the most important papers involved in URVA is the introduction of Diabatic Mode Ordering(DMO) procedure which has now been widely used in several projects within CATCO group\cite{dmopaper}.





%dmopaper





 % Before we start 

\lhead{\emph{Chapter 2}}

\chapter{Input description}

There are two major types of input that are allowed in the input file. 

\begin{itemize}
   \setlength{\itemsep}{0pt}
   \setlength{\parskip}{0pt}
   \setlength{\parsep}{0pt} 
    \item Keyword input
    \item Section input

\end{itemize}

Besides the input file that can be edited by the users, pURVA  needs  input data files which record  necessary information along the IRC path. 


\section{Input Browsing Data Files}

The browsing files are in a compact form generated from any quantum chemistry package, e.g. Gaussian, that can do a reaction path calculation (i.e. IRC). 

There are currently two types or versions of browsing file. The first one can be generated with Gaussian.

In Gaussian package, the functionality to generate the first version of browsing files has been implemented in 09. D  and later versions, where we can specify IOp(1/45D=1000000) to ask the program to dump the  {URVA} input file.

For the other version, which is a newer version of browsing file, only COLOGNE program (modified Gaussian links) could generate it by specifying  IOp(1/169=1). [Please confirm with Niraj Verma for the latest implementation.]

%To make  \texttt{standalone URVA} work, the browsing file must take the filename of  \texttt{IRC.forward} for old version of browsing file or \texttt{for.urv} for new version of browsing file.



An old version of browsing file \texttt{IRC.forward} looks like this:

%%%%
\lstset{basicstyle=\ttfamily\footnotesize,breaklines=true}
\lstinputlisting{IRC.Forward}
%%%%%%%




This is all browsing information for one point along the reaction path, a complete browsing file contains many points. 

\lstset{basicstyle=\ttfamily\footnotesize,breaklines=true}
The line starting with \texttt{BEGIN} is the starting line for this point. Followed by a negative floating number which is the reaction coordinate(parameter) for the current point.

The first column of numbers in \texttt{IAnZ} section is the atomic number for each atoms.

The first number in \texttt{NAtom} section is the number of atoms. 

The  \texttt{CC} section is the cartesian coordinates, the unit here is Bohr instead of Angstrom. It takes the dimension of \texttt{3*NAtom}.

The  \texttt{FX\_ZMat\_Orientation} section is the Gradient information. It takes the dimension of \texttt{3*NAtom}.

The  \texttt{FFX\_ZMat\_Orientation} section is the Hessian(Force constant) information. It takes the dimension of \texttt{3*NAtom*(3*NAtom-1)/2+3*NAtom} in a lower-triangular form.

The  \texttt{IPoCou, Energy, XXIRC} section gives the label of points along the reaction path, electronic structure energy in Hartree and reaction coordinate(parameter).

%\end{section}

A new version of browsing file \texttt{for.urv} looks like this:
\lstset{basicstyle=\ttfamily\footnotesize,breaklines=true}
\lstinputlisting{for.urv}

To distinguish from old browsing file, the first line is \texttt{BEGIN (no Hessian)}, indicating that this browsing file does not contains Hessian information and is a new browsing file.

The second section of \texttt{Natoms,NatomQ} gives the total number of atoms in the system and number of atoms in the QM part. If the latter is smaller than the previous, it means this is an QMMM calculation. (in gaussian, it is an ONIOM calculation.)

The \texttt{Atomic mass} section gives all atomic masses for atoms in the system.

The \texttt{CC} section gives the cartesian coordinates information, the unit here is Bohr instead of Angstrom. It takes the dimension of \texttt{3*NAtom}.

The \texttt{Tangent} section gives the mass-weighted path direction vector which has been mass-weighted. It takes the dimension of \texttt{3*NAtom}.

The \texttt{Curvature} section gives the mass-weighted curvature vector which has been mass-weighted. It takes the dimension of \texttt{3*NAtom}.

The \texttt{IPoCou, Energy, XXIRC} section firstly gives the point label along the reaction path, which can range from a very negative integer to a very positive integer. The second number is the electronic structure energy in Hartree. The last number is the reaction coordinate(parameter) which can range from a very negative floating number to a very positive floating number.

%The \texttt{Freq Numbers(QM, ALL)} section gives the number of QM part frequencies and number of frequencies for the whole system.

%The \texttt{QM Frequency} section gives the vibrational frequency values of the QM part, which takes the unit of $cm^{-1}$.

%The \texttt{All Frequency} section gives the vibrational frequency values for the whole system, which takes the unit of $cm^{-1}$. And this section contains the last section of \texttt{QM Frequency}.

%The \texttt{QM Imaginary Freq No.} section gives the number of imaginary frequencies with regard to the QM part.

%The \texttt{QM Imaginary Normal Modes} section gives the normal mode vector of imaginary frequency vibrations in the QM part, the length of this section is \texttt{M*3*NAtom}, in which  \texttt{M} is the number of imaginary frequencies with regard to the QM part





\section{Keyword input}
Just for convenience, keyword input is often written before section input. The format of keyword input line is:

\texttt{\textcolor{red}{@\textit{keyword\_name}} = \textcolor{blue}{[keyword\_value]} }

The \texttt{@} symbol should be in the first column. No space is allowed after it. On both sides of = sign, it should be space. There might be several optional keyword values available, however, only one option is accepted.

\subsection{\texttt{@DATAFILETYPE} keyword }

This keywords specifies the format of input data source file for URVA analysis. 

\texttt{\textcolor{red}{@DATAFILETYPE} = \textcolor{blue}{old/new/xyz}}

\textcolor{blue}{\texttt{old}}: The input data source file is generated by Gaussian package by setting corresponding IOp(1/45). This type of data contains most complete information.

\textit{NOTE: If the data file is generated by Gaussian with version number lower than 16.A, all floating numbers should be converted from "D" into "E" format. }

\textcolor{blue}{\texttt{new}}: The input data source file is generated by a modified version of Gaussian package. This type of data has no Hessian and gradient stored. 

\textcolor{blue}{\texttt{xyz}}: XYZ file containing the Cartesian coordinates of multiple snapshots.


\subsection{\texttt{@DATAFILEPATH} keyword }
This keyword specifies the path of the input data file.

\texttt{\textcolor{red}{@DATAFILEPATH} = \textcolor{blue}{"../path/to/data/file"} }

The quotation marks should be included.


\subsection{\texttt{@ENERGY} keyword }
This keyword specifies whether SCF energy and its first and second derivatives will be calculated.


\texttt{\textcolor{red}{@ENERGY} = \textcolor{blue}{on/off}  }










\subsection{\texttt{@PARM} keyword }
This keyword specifies the way to deal with internal coordinates parameters provided by user.

\texttt{\textcolor{red}{@PARM} = \textcolor{blue}{No/GeomOnly/All}  }

\textcolor{blue}{\texttt{No}}: Do nothing with regard to these internal coordinates specifications.

\textcolor{blue}{\texttt{GeomOnly}}: Only calculate the value of these internal coordinates. 


\textcolor{blue}{\texttt{All}}: Besides the value of internal coordinates, other properties related to these internal coordinates will be calculated.



\subsection{\texttt{@VIBRATION} keyword }
This keyword specifies whether or not to do normal mode analysis.

\texttt{\textcolor{red}{@VIBRATION} = \textcolor{blue}{on/off}  }

If the keyword value is set to \texttt{\textcolor{blue}{on}}, the \texttt{\textcolor{red}{@DATAFILETYPE}} must be set to \texttt{\textcolor{blue}{old}}.
% and \texttt{\textcolor{red}{@DIRCURV}} must be set to \texttt{\textcolor{blue}{on}}.


\subsection{\texttt{@DIRCURV} keyword}
This keyword decides whether or not to calculate reaction path direction $\boldsymbol {\eta(s)}$ and curvature $\boldsymbol {\kappa(s)}$.

\texttt{\textcolor{red}{@DIRCURV} = \textcolor{blue}{on/off}  }

If the keyword value is set to  \texttt{\textcolor{blue}{on}}, the \texttt{\textcolor{red}{@DATAFILETYPE}} must be set to \texttt{\textcolor{blue}{old}} or \texttt{\textcolor{blue}{new}}. 




\subsection{\texttt{@AVAM} keyword}
This keyword specifies whether or not to calculate the adiabatic mode coupling coefficient $\boldsymbol{ A_{n,s}(s)}$.


\texttt{\textcolor{red}{@AVAM} = \textcolor{blue}{on/off}  }

If the keyword value is set to \texttt{\textcolor{blue}{on}}, the \texttt{\textcolor{red}{@DATAFILETYPE}} must be set to \texttt{\textcolor{blue}{old}}, 
the \texttt{\textcolor{red}{@PARM}} must be set to \texttt{\textcolor{blue}{All}},
the \texttt{\textcolor{red}{@VIBRATION}} must be set to \texttt{\textcolor{blue}{on}} and the \texttt{\textcolor{red}{@DIRCURV}} must be set to \texttt{\textcolor{blue}{on}}




\subsection{\texttt{@CURVCPL} keyword}
This keyword specifies whether or not to calculate the curvature coupling coefficient $\boldsymbol{B_{\mu,s}(s)}$.

\texttt{\textcolor{red}{@CURVCPL} = \textcolor{blue}{on/off}  }

If the keyword value is set to \texttt{\textcolor{blue}{on}},
the \texttt{\textcolor{red}{@DATAFILETYPE}} must be set to \texttt{\textcolor{blue}{old}}, 
the \texttt{\textcolor{red}{@VIBRATION}} must be set to \texttt{\textcolor{blue}{on}}, 
and the \texttt{\textcolor{red}{@DIRCURV}} must be set to \texttt{\textcolor{blue}{on}}.


\subsection{\texttt{@CORIOLIS} keyword}
This keyword specifies whether or not to calculate the Coriolis mode-mode coupling coefficient $\boldsymbol{B_{\mu,\nu}(s)}$.

\texttt{\textcolor{red}{@CORIOLIS} = \textcolor{blue}{on/off}  }

If the keyword value is set to \texttt{\textcolor{blue}{on}},
the \texttt{\textcolor{red}{@DATAFILETYPE}} must be set to \texttt{\textcolor{blue}{old}} and the \texttt{\textcolor{red}{@VIBRATION}} must be set to \texttt{\textcolor{blue}{on}}.




\subsection{\texttt{@ADIABFC} keyword}
This keyword specifies whether or not to calculate adiabatic force constant $\mathbf{k^a}$.

\texttt{\textcolor{red}{@ADIABFC} = \textcolor{blue}{on/off}  }

If the keyword value is set to \texttt{\textcolor{blue}{on}},
the \texttt{\textcolor{red}{@DATAFILETYPE}} must be set to \texttt{\textcolor{blue}{old}} and the \texttt{\textcolor{red}{@PARM}} must be set to \texttt{\textcolor{blue}{All}}.





\section{Section input}
Section input is used when multiple parameters need to be read in, the format of the section input is:

\texttt{
\textcolor{Purple}{SECTION\_NAME}}

\texttt{
\textcolor{red}{parameter line\_1}}

\texttt{
\textcolor{red}{parameter line\_2}}

\texttt{
\textcolor{red}{...}}

\texttt{
\textcolor{Purple}{END SECTION\_NAME}}


\subsection{\texttt{TITLE} section  }
This section accepts remarks provided by user. The content will be displayed in standard output.

\texttt{
\textcolor{Purple}{TITLE}}

\texttt{
\textcolor{red}{Please put remarks here.}}

\texttt{
\textcolor{red}{Multiple lines are accepted.}}

\texttt{
\textcolor{Purple}{END TITLE}}

This section is quite useful to take note of the parameters we use for URVA calculations.



\subsection{\texttt{PARAMETER} section }
This section contains the internal coordinates specifications provided by the user. Different types of internal coordinates including ring coordinates are acceptable.

Bond length, bond angle, dihedral angle, out-of-plane angle, pyramidalization angle, ring puckering amplitude, ring puckering phase angle, ring deformation amplitude and ring deformation phase angle are supported.   


\texttt{
\textcolor{Purple}{PARAMETER}}

\texttt{
\textcolor{red}{\textit{Internal coordinate specification}}}

\texttt{
\textcolor{Purple}{END PARAMETER}}

Bond length:

\texttt{
\textcolor{red}{
std $N_1$ $N_2$ : "bond\_name"
}}

Bond angle:

\texttt{
\textcolor{red}{
std $N_1$ $N_2$ $N_3$ : "angle\_name"
}}

Dihedral angle:

\texttt{
\textcolor{red}{
std $N_1$ $N_2$ $N_3$ $N_4$ : "dihedral\_name"
}}

Out of plane angle(the angle between the bond length $N_1$-$N_2$ and the plane $N_2$-$N_3$-$N_4$):

\texttt{
\textcolor{red}{
oop $N_1$ $N_2$ $N_3$ $N_4$ : "out\_of\_plane\_name"
}}


Pyramidalization angle(the angle $\theta_P$ is related to the three bond angles $N_2$-$N_1$-$N_3$, $N_3$-$N_1$-$N_4$, $N_4$-$N_1$-$N_2$):


\texttt{
\textcolor{red}{
pyr $N_1$ $N_2$ $N_3$ $N_4$ : "pyramidalization\_angle\_name"
}}


Radius of planar reference ring($R$)($N_{ring}$: number of ring atoms):


\texttt{
\textcolor{red}{
ring $N_{ring}$ - ( $N_1$ $N_2$ ... $N_{atoms}$ ) -[0 0]: "ring\_breathing\_name"
}}

Planar deformation amplitude($t_n$)(n=1$\sim$$N_{ring}-2$):

\texttt{
\textcolor{red}{
ring $N_{ring}$ - ( $N_1$ $N_2$ ... $N_{atoms}$ ) -[1 n]: "deformation\_amplitude\_name"
}}

Planar deformation phase angle($\tau_n$)(n=1$\sim$$N_{ring}-2$):

\texttt{
\textcolor{red}{
ring $N_{ring}$ - ( $N_1$ $N_2$ ... $N_{atoms}$ ) -[2 n]: "deformation\_phase\_angle\_name"
}}


Puckering amplitude($q_n$)(n=2$\sim$($N_{ring}-1$)/2 for odd $N_{ring}$ or 2$\sim$$N_{ring}$/2 for even $N_{ring}$):

\texttt{
\textcolor{red}{
ring $N_{ring}$ - ( $N_1$ $N_2$ ... $N_{atoms}$ ) -[3 n]: "puckering\_amplitude\_name"
}}


Puckering phase angle($\phi_n$)(n=2$\sim$($N_{ring}-1$)/2 for odd $N_{ring}$ or 2$\sim$$N_{ring}$/2-1 for even $N_{ring}$):


\texttt{
\textcolor{red}{
ring $N_{ring}$ - ( $N_1$ $N_2$ ... $N_{atoms}$ ) -[4 n]: "puckering\_phase\_angle\_name"
}}


\subsection{\texttt{CURVCOR} section }
The CURVCOR interface will be activated if this section is found.

For most situations, it is usually enough for 
\texttt{\textcolor{red}{$N_l$}} and \texttt{\textcolor{red}{$N_r$}} to take the value of 25.


\texttt{
\textcolor{Purple}{CURVCOR}}

\texttt{
\textcolor{red}{Ln = $N_l$}}


\texttt{
\textcolor{red}{Rn = $N_r$}}


\texttt{
\textcolor{Purple}{END CURVCOR}}



\subsection{\texttt{AUTOSMTH} section }
The AUTOSMTH interface will be activated if this section is found.

AUTOSMTH interface requires the activation of CURVCOR interface.

\texttt{\textcolor{red}{$\delta s$}} is the stepsize of mass-weighted IRC with the unit of amu$^{1/2}$-Bohr.

Using the value of 3 is usually enough for 
\texttt{\textcolor{red}{$N_l$}} and \texttt{\textcolor{red}{$N_r$}}.

\texttt{\textcolor{red}{$t$}} is a cut-off for second derivative of smoothened curve. Increase it when necessary. Recommended value: 2.5.



\texttt{
\textcolor{Purple}{AUTOSMTH}}

\texttt{
\textcolor{red}{StepSize = $\delta s$}}

\texttt{
\textcolor{red}{Ln = $N_l$}}

\texttt{
\textcolor{red}{Rn = $N_r$}}

\texttt{
\textcolor{red}{d2ythresh = $t$}}


\texttt{
\textcolor{Purple}{END AUTOSMTH}}


\subsection{\texttt{RMSPK} section }
The RMSPK interface will be activated if this section is found.

RMSPK interface requires the activation of AUTOSMTH.

Any points in the curvature plot having the value larger than \texttt{\textcolor{red}{$k$}} will be left out as spike.

The value of \texttt{\textcolor{red}{$p$}} ranges from 0.5 to 1.0 as a percentage number. Any points leading to consecutive difference larger than the percentile of \texttt{\textcolor{red}{$p$}} will be labeled as spike condidates. Recommened value: 0.85.

Gradient check threshold \texttt{\textcolor{red}{$g$}} is used to filter out normal points from spike candidates. Recommended value: 1.2.



\texttt{
\textcolor{Purple}{RMSPK}}

\texttt{
\textcolor{red}{CutHigh = $k$}}

\texttt{
\textcolor{red}{Percentage = $p$}}


\texttt{
\textcolor{red}{GradRatio = $g$}}


\texttt{
\textcolor{Purple}{END RMSPK}}




\subsection{\texttt{DMO} section }
If this section input is not found, default parameter values will be used. 


\texttt{\textcolor{red}{$s_{max}$}} is an overlap threshold after each mode reordering step. If the overlap criteria of \texttt{\textcolor{red}{$s_{max}$}} could not be reached, the criteria will be reduced to \texttt{\textcolor{red}{$s_{min}$}} gradually. Recommend values for \texttt{\textcolor{red}{$s_{max}$}} and \texttt{\textcolor{red}{$s_{min}$}}: 0.990 and 0.890.

If local difficulty is encountered, linear interpolation will be adopted, space between two consecutive points will be divided into \texttt{\textcolor{red}{$N_{min}$}} pieces. If the difficulty is still not solved, \texttt{\textcolor{red}{$N_{min}$}} will be increased up to \texttt{\textcolor{red}{$N_{max}$}}. Recommended values for \texttt{\textcolor{red}{$N_{min}$}} and \texttt{\textcolor{red}{$N_{max}$}}: 30 and 200.


If the DMO could not get through for a specific point due to the following reasons:

\begin{itemize}
   \item Change of symmetry of reaction complex, e.g. linear $\rightarrow$ non-linear 
   \item Discontinuity of reaction path
   \item Failure of reaction path following close to local minimum region 
\end{itemize}

one solution to circumvent this problem is to calculate and re-order the vibrational frequencies for a specific region of reaction path. This function could be activated by setting \texttt{\textcolor{red}{$IO_{cut}$}} to 1. In this way, the reaction path with its $\mathbf{s}$ value ranging from \texttt{\textcolor{red}{$s_{start}$}} to \texttt{\textcolor{red}{$s_{end}$}} will have vibrational frequencies calculated.

In some situations, due to the innate difficulty of path following algorithm, the DMO might fail at the transition state(TS) point. And also the first point off TS point in either forward or reverse direction might also lead to problems. In order to remediate this problem, we can skip a few points in that region by setting \texttt{\textcolor{red}{$IO_{skip}$}} to 1. If one point off the TS point in reverse(or forward) direction also needs to be skipped, \texttt{\textcolor{red}{$N_{left}$}}( or \texttt{\textcolor{red}{$N_{right}$}}) should be set to 1. 




\texttt{
\textcolor{Purple}{DMO}}

\texttt{
\textcolor{red}{Sthresh = $s_{max}$}}

\texttt{
\textcolor{red}{Slowest = $s_{min}$}}


\texttt{
\textcolor{red}{Np = $N_{min}$}}


\texttt{
\textcolor{red}{NMax = $N_{max}$}}


\texttt{
\textcolor{red}{Cut = $IO_{cut}$}}

\texttt{
\textcolor{red}{CutA = $s_{start}$}}

\texttt{
    \textcolor{red}{CutB = $s_{end}$}}


\texttt{
\textcolor{red}{Skip = $IO_{skip}$}}

\texttt{
\textcolor{red}{SkipA = $N_{left}$}}

\texttt{
\textcolor{red}{SkipB = $N_{right}$}}

\texttt{
\textcolor{Purple}{END DMO}}


 % Input description 

\lhead{\emph{Chapter 3}}

\chapter{Output description}

In pURVA, results are all written to external files instead of standard output. All output files have the suffix of ``.csv'' or ``.dat''. The execution of pURVA will abort if a result file with a duplicated name is found in the current folder. Make sure that current directory is cleaned up before execution.


\section{Energy and derivatives}

Usually the Self-Consistent Field(SCF) energy is calculated and used to construct the potential energy surface along the reaction path.

In order to check this value, the \texttt{\textcolor{red}{@DATAFILETYPE}} must be set to \texttt{\textcolor{blue}{old}} or \texttt{\textcolor{blue}{new}} and \texttt{\textcolor{red}{@ENERGY}} must be set to \texttt{\textcolor{blue}{on}}.

The unit of SCF energy is Hartree as one of the atomic units(a.u.). In order to calculate first and second derivatives of SCF energy against reaction coordinate/parameter $\mathbf{s}$, cubic spline fitting is used. For the second derivative of SCF energy, the region between $\mathbf{s}=-0.1$ and $\mathbf{s}=+0.1$ is predicted via cubic spline fitting from the information outside this region.

\textit{NOTE: 1 Hartree = 627.509 474 kcal/mol}

Output files:
\begin{itemize}
    \item \texttt{\textcolor{blue}{energy.csv}}
    
    SCF energy vs. $\mathbf{s}$
    
    \item \texttt{\textcolor{blue}{energy\_1\_d.csv}}
    
    First derivative of SCF energy vs. $\mathbf{s}$
    
    \item \texttt{\textcolor{blue}{energy\_2\_d.csv}}    
    
    Second derivative of SCF energy vs. $\mathbf{s}$    
    
\end{itemize}



\section{Internal coordinates}
The value of user-defined internal coordinates could be calculated. All types of internal coordinates described in section 2.2.2 are supported. 

In order to have this result, \texttt{\textcolor{red}{@PARM}} must be set to \texttt{\textcolor{blue}{GeomOnly}} or \texttt{\textcolor{blue}{All}}.

The unit of printed internal coordinates is atomic unit with bohr for distance and radian for angles.

\textit{NOTE: 1 Bohr = 0.529177 $\AA$; 1 rad = 57.295 8 $^{\circ}$ }


Output file:
\begin{itemize}
    \item \texttt{\textcolor{blue}{q$\_$n.csv}}
\end{itemize}






\section{Decomposition of reaction path direction and curvature into internal coordinates}


In order to have this result, \texttt{\textcolor{red}{@PARM}} must be set to  \texttt{\textcolor{blue}{All}}.

Output files:
\begin{itemize}
    \item \texttt{\textcolor{blue}{eta-q\_n.csv}}
    
    Decomposition of reaction path direction into internal coordinates
    
    \item \texttt{\textcolor{blue}{kappa-q\_n.csv}}
    
    Decomposition of reaction path curvature into internal coordinates
    
\end{itemize}


\section{Generalized vibrational frequency}
For any point on reaction path, we could have $3N - K - 1$ vibrations, in which $K$ is the total number of translations and rotations. In pURVA, $K$ takes the value of 6 which excludes the possibility of analysis of reactions like H$_2$ + H $\rightarrow$ H + H$_2$ where the whole reaction complex stays in a linear geometry.

In order to have this result, \texttt{\textcolor{red}{@VIBRATION}} must be set to  \texttt{\textcolor{blue}{on}}.

Output file:
\begin{itemize}
    \item \texttt{\textcolor{blue}{freq\_dmo.csv}}
    
    Generalized vibrational frequencies vs. $\mathbf{s}$

\end{itemize}

\textit{NOTE: Unit of frequencies is $cm^{-1}$.}


\section{Scalar curvature}

The original scalar curvature calculated without correction around the TS region and spike removal will be written to file \texttt{\textcolor{blue}{originalkappa.dat}}.

In order to have this result, \texttt{\textcolor{red}{@DIRCURV}} must be set to  \texttt{\textcolor{blue}{on}}.


If the CURVCOR and AUTOSMTH modules are used, the corrected curvature data will be written to \texttt{\textcolor{blue}{merged.dat}}.

If RMSPK module is also used, the curvature data after spike removal will be written to \texttt{\textcolor{blue}{merged-nospk.dat}}.




\section{Adiabatic force constant}
The adiabatic force constant of chemical bonds between two atoms along the reaction path will be written to \texttt{\textcolor{blue}{adiabfc-ka.csv}}.

In order to have this result, \texttt{\textcolor{red}{@ADIABFC}} must be set to  \texttt{\textcolor{blue}{on}}.


In some situations, there might be noise in the result. These noise regions could be nicely removed via cubic spline fitting.

\textit{NOTE: Only result of bond length between 2 atoms could make sense.}




 % Output description

\lhead{\emph{Chapter 4}}
\chapter{Examples}


\section{Example 1. HCN $\rightarrow$ HNC isomerization}

\begin{tcolorbox}

\begingroup\singlespacing
\begin{verbatim}
@DATAFILETYPE = old   
@PARM = All 
@VIBRATION = on
@DIRCURV = on  
@AVAM = off  
@CURVCPL = off  
@CORIOLIS = off  
@ENERGY = on
@ADIABFC = off

@DATAFILEPATH = "./examples/hcn/IRC.browse"

TITLE
 HCN Reaction test job
 Frequency calculation from -3.09985216909 to 3.97984648135
END TITLE

PARAMETER
std 2 3 : bond-NH
std 1 3 : bond-CH
END PARAMETER

DMO
Sthresh = 0.990
Slowest = 0.890
Np = 30
NMax = 200
Cut = 1
CutA = -3.09985216909
CutB = 3.97984648135
END DMO

\end{verbatim}
\endgroup

\end{tcolorbox}



\section{Example 2. CH$_3$ + H$_2$ $\rightarrow$ CH$_4$ + H}

The whole calculation may take up to 15 minutes.


\begin{tcolorbox}

\begingroup\singlespacing
\begin{verbatim}
@DATAFILETYPE = old   
@PARM = All 
@VIBRATION = on
@DIRCURV = on  
@AVAM = off  
@CURVCPL = off  
@CORIOLIS = off  
@ENERGY = on
@ADIABFC = off

@DATAFILEPATH = "./examples/ch3h2/IRC.forward.2"

TITLE
 CH3+H2 Reaction test job
END TITLE

PARAMETER
std 2 3 : bond-NH
std 1 3 : bond-CH
END PARAMETER

DMO
Sthresh = 0.990
Slowest = 0.890
Np = 30
NMax = 200
Cut = 0
END DMO

\end{verbatim}
\endgroup

\end{tcolorbox}



\section{Example 3. Gold catalysis Step-1}
For this reaction complex, we do three URVA analysis runs. In the first run, basic information including energy, internal coordinates are calculated. In the second run, scalar curvature is corrected in its TS region and spikes are also removed. In the last run, vibrational frequencies are calculated for three segments of the reaction path due to several local difficulties around TS point.




\subsection{First run}

\begin{tcolorbox}

\begingroup\singlespacing
\begin{verbatim}
@DATAFILETYPE = old   
@PARM = GeomOnly 
@VIBRATION = off
@DIRCURV = off  
@AVAM = off  
@CURVCPL = off  
@CORIOLIS = off  
@ENERGY = on
%%@ADIABFC = off

@DATAFILEPATH = "./examples/gold/IRC.browse"
@BASEPATH = "/path/to/pURVA/folder"


TITLE
 Gold catalysis step 1 test job - first run
END TITLE

PARAMETER
std 1 16 : bond-C1C16
std 5 16 : bond-O5C16
std 1 10 : bond-C1C10
std 2 5 :  bond-C2O5
std 2 4 :  bond-C2O4
END PARAMETER

\end{verbatim}
\endgroup

\end{tcolorbox}


\subsection{Second run}

\begin{tcolorbox}

\begingroup\singlespacing
\begin{verbatim}
@DATAFILETYPE = old   
@PARM = All 
@VIBRATION = off
@DIRCURV = on
@AVAM = off  
@CURVCPL = off  
@CORIOLIS = off  
@ENERGY = off
@ADIABFC = off

@DATAFILEPATH = "./examples/gold/IRC.browse"

TITLE
 Gold catalysis step 1 test job - second run
END TITLE

PARAMETER
std 1 16 : bond-C1C16
std 5 16 : bond-O5C16
std 1 10 : bond-C1C10
std 2 5 :  bond-C2O5
std 2 4 :  bond-C2O4
END PARAMETER

CURVCOR
Ln = 25
Rn = 25
END CURVCOR

AUTOSMTH
StepSize = 0.03
Ln = 3
Rn = 3
d2ythresh = 2.4
END AUTOSMTH

RMSPK
CutHigh = 20.0
Percentage = 0.85
GradRatio = 1.2
END RMSPK

\end{verbatim}
\endgroup

\end{tcolorbox}




\subsection{Third run}

This run may take up to 25 minutes.


\begin{tcolorbox}

\begingroup\singlespacing
\begin{verbatim}
@DATAFILETYPE = old   
@PARM = No
@VIBRATION = on
@DIRCURV = off
@AVAM = off  
@CURVCPL = off  
@CORIOLIS = off  
@ENERGY = off
@ADIABFC = off

@DATAFILEPATH = "./examples/gold/IRC.browse"

TITLE
 Gold catalysis step 1 test job - third run
END TITLE

DMO
Sthresh = 0.980
Slowest = 0.880
Np = 5
NMax = 80
Cut = 0
END DMO

\end{verbatim}
\endgroup

\end{tcolorbox}


 % Examples

%\input{Chapters/Chapter5} % Experiment 2

%\input{Chapters/Chapter6} % Results and Discussion

%\input{Chapters/Chapter7} % Conclusion

%% ----------------------------------------------------------------
% Now begin the Appendices, including them as separate files

\addtocontents{toc}{\vspace{2em}} % Add a gap in the Contents, for aesthetics

%\appendix % Cue to tell LaTeX that the following 'chapters' are Appendices

%\chapter{An Appendix}

Lorem ipsum dolor sit amet, consectetur adipiscing elit. Vivamus at pulvinar nisi. Phasellus hendrerit, diam placerat interdum iaculis, mauris justo cursus risus, in viverra purus eros at ligula. Ut metus justo, consequat a tristique posuere, laoreet nec nibh. Etiam et scelerisque mauris. Phasellus vel massa magna. Ut non neque id tortor pharetra bibendum vitae sit amet nisi. Duis nec quam quam, sed euismod justo. Pellentesque eu tellus vitae ante tempus malesuada. Nunc accumsan, quam in congue consequat, lectus lectus dapibus erat, id aliquet urna neque at massa. Nulla facilisi. Morbi ullamcorper eleifend posuere. Donec libero leo, faucibus nec bibendum at, mattis et urna. Proin consectetur, nunc ut imperdiet lobortis, magna neque tincidunt lectus, id iaculis nisi justo id nibh. Pellentesque vel sem in erat vulputate faucibus molestie ut lorem.

Quisque tristique urna in lorem laoreet at laoreet quam congue. Donec dolor turpis, blandit non imperdiet aliquet, blandit et felis. In lorem nisi, pretium sit amet vestibulum sed, tempus et sem. Proin non ante turpis. Nulla imperdiet fringilla convallis. Vivamus vel bibendum nisl. Pellentesque justo lectus, molestie vel luctus sed, lobortis in libero. Nulla facilisi. Aliquam erat volutpat. Suspendisse vitae nunc nunc. Sed aliquet est suscipit sapien rhoncus non adipiscing nibh consequat. Aliquam metus urna, faucibus eu vulputate non, luctus eu justo.

Donec urna leo, vulputate vitae porta eu, vehicula blandit libero. Phasellus eget massa et leo condimentum mollis. Nullam molestie, justo at pellentesque vulputate, sapien velit ornare diam, nec gravida lacus augue non diam. Integer mattis lacus id libero ultrices sit amet mollis neque molestie. Integer ut leo eget mi volutpat congue. Vivamus sodales, turpis id venenatis placerat, tellus purus adipiscing magna, eu aliquam nibh dolor id nibh. Pellentesque habitant morbi tristique senectus et netus et malesuada fames ac turpis egestas. Sed cursus convallis quam nec vehicula. Sed vulputate neque eget odio fringilla ac sodales urna feugiat.

Phasellus nisi quam, volutpat non ullamcorper eget, congue fringilla leo. Cras et erat et nibh placerat commodo id ornare est. Nulla facilisi. Aenean pulvinar scelerisque eros eget interdum. Nunc pulvinar magna ut felis varius in hendrerit dolor accumsan. Nunc pellentesque magna quis magna bibendum non laoreet erat tincidunt. Nulla facilisi.

Duis eget massa sem, gravida interdum ipsum. Nulla nunc nisl, hendrerit sit amet commodo vel, varius id tellus. Lorem ipsum dolor sit amet, consectetur adipiscing elit. Nunc ac dolor est. Suspendisse ultrices tincidunt metus eget accumsan. Nullam facilisis, justo vitae convallis sollicitudin, eros augue malesuada metus, nec sagittis diam nibh ut sapien. Duis blandit lectus vitae lorem aliquam nec euismod nisi volutpat. Vestibulum ornare dictum tortor, at faucibus justo tempor non. Nulla facilisi. Cras non massa nunc, eget euismod purus. Nunc metus ipsum, euismod a consectetur vel, hendrerit nec nunc.	% Appendix Title

%\input{Appendices/AppendixB} % Appendix Title

%\input{Appendices/AppendixC} % Appendix Title

\addtocontents{toc}{\vspace{2em}}  % Add a gap in the Contents, for aesthetics
\backmatter

%% ----------------------------------------------------------------
\label{References}
%\lhead{References}  % Change the left side page header to "Bibliography"
\bibliographystyle{unsrtnat}  % Use the "unsrtnat" BibTeX style for formatting the Bibliography
\bibliography{Bibliography}  % The references (bibliography) information are stored in the file named "Bibliography.bib"

\end{document}  % The End
%% ----------------------------------------------------------------